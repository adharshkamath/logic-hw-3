\documentclass[12pt,letterpaper, onecolumn]{exam}
\usepackage{amsmath}
\usepackage{amssymb}
\usepackage{listings}
\usepackage{hyperref}
\usepackage{xcolor}
\usepackage{bookmark}
\usepackage{graphicx}
\newcommand{\link}[1]{{\color{blue}\href{#1}{#1}}}
\usepackage{pythonhighlight}
\usepackage[a4paper,lmargin=30pt, rmargin=50pt, tmargin=0.65in]{geometry}  %For centering solution box
% \chead{\hline} % Un-comment to draw line below header
\thispagestyle{empty}   %For removing header/footer from page 1

\begin{document}

\begingroup
\centering
\LARGE CS 474\\
\large Homework 3 \\[0.5em]
\endgroup
\begingroup
\normalsize \quad\quad\quad Name: Adharsh Kamath \quad\quad\quad \quad\quad\quad \quad\quad\quad \quad\quad\quad \quad  NetID: ak128 \par\
\endgroup
\rule{17cm}{0.4pt}
\pointsdroppedatright   %Self-explanatory
\printanswers
\renewcommand{\solutiontitle}{\noindent\textbf{Soln:}\enspace}
\newcommand{\cheading}[1]{{\underline{\textit{#1}}}}

\renewcommand{\questionshook}{%
	\setlength{\leftmargin}{18pt}%
	\setlength{\labelwidth}{-\labelsep}%
}
\qformat{\underline{Problem \thequestion}}
\begin{questions}
	\question[]
	\solutiontitle
	We are given three models with signatures $ (\text{U}, 0, 1, +, *) $ where $ \text{U} $ is $\mathbb{N}$, $\mathbb{Q}$ and $\mathbb{R}$.

	\textbf{(a)} \\
	For each formula given below, we shall see if the formula is true or false over $\mathbb{N}$, $\mathbb{Q}$ and $\mathbb{R}$. \\ \\
	1. $ \exists y. ( y * y = 1 + 1 ) $ \\ \\
	\begin{tabular}{ |c|c|c| }
		\hline
		$\mathbb{N}$ & $\mathbb{Q}$ & $\mathbb{R}$ \\
		\hline
		False & False & True \\
		\hline
	\end{tabular} \\ \\
	This formula is true over $\mathbb{R}$ since there is a real number ($ \sqrt{2} $) which satisfies the formula. However, the same is not true
	for $\mathbb{N}$ and $\mathbb{Q}$. \\

	2. $ \forall x. \exists y. (x + y = 0) $ \\ \\
	\begin{tabular}{ |c|c|c| }
		\hline
		$\mathbb{N}$ & $\mathbb{Q}$ & $\mathbb{R}$ \\
		\hline
		False & True & True \\
		\hline
	\end{tabular} \\ \\
	This formula is true over $\mathbb{Q}$ and $\mathbb{R}$ since for any $x \in \mathbb{Q}$ (or $\mathbb{R}$), there exists a 
	$y \in \mathbb{Q}$ (or $\mathbb{R}$) such that $x + y = 0$. (We can take $y = -x$ when $ x \neq 0 $ and $ y = 0 $ otherwise).
	However, this is not true for $\mathbb{N}$ since there is no $y \in \mathbb{N}$ such that $x + y = 0$ for any $x \in \mathbb{N}$. \\

	3. $ \forall x. \forall y. (\neg(y = 0) \Rightarrow (\exists z. x * y = x + z)) $ \\ \\
	\begin{tabular}{ |c|c|c| }
		\hline
		$\mathbb{N}$ & $\mathbb{Q}$ & $\mathbb{R}$ \\
		\hline
		True & True & True \\
		\hline
	\end{tabular} \\ \\
	This formula is true over $\mathbb{N}$, $\mathbb{Q}$ and $\mathbb{R}$ since for any $x, y \in \mathbb{N}$ (or $\mathbb{Q}$ or $\mathbb{R}$),
	we can find a $z \in \mathbb{N}$ (or $\mathbb{Q}$ or $\mathbb{R}$) such that $x * y = x + z$. (We can take $z = x * (y - 1)$, and still ensure
	$ z \in \mathbb{N} $ (or $\mathbb{Q}$ or $\mathbb{R}$). This is possible since the premise of the implication is that $y \neq 0$). \\

	4. $ \exists x. \exists y. (x + 1 = 0 \land y * y = x) $ \\ \\
	\begin{tabular}{ |c|c|c| }
		\hline
		$\mathbb{N}$ & $\mathbb{Q}$ & $\mathbb{R}$ \\
		\hline
		False & False & False \\
		\hline
	\end{tabular} \\ \\
	This formula is false over $\mathbb{N}$, $\mathbb{Q}$ and $\mathbb{R}$ since there is no $x, y \in \mathbb{N}$ (or $\mathbb{Q}$ or $\mathbb{R}$)
	such that $x + 1 = 0$ and $y * y = x$. This formula can evaluate to true only if $x = -1$ and $y = \pm \sqrt{-1}$, which is possible 
	only in $\mathbb{C}$ (complex numbers) and not in $\mathbb{N}$, $\mathbb{Q}$ or $\mathbb{R}$. \\

	\textbf{(b)} \\
	Task 1: \\
	We need to write a formula $ gt_{\mathbb{N}}(x, y) $ that is true precisely when $ x > y $ for any $ x, y \in \mathbb{N} $. \\
	We can write the formula as $ gt_{\mathbb{N}}(x, y) = \exists z. (z \neq 0 \land x = y + z) $. \\

	Task 2: \\
	We need to write a formula $ gt_{\mathbb{R}}(x, y) $ that is true precisely when $ x > y $ for any $ x, y \in \mathbb{R} $. \\
	We can write the formula as $ gt_{\mathbb{R}}(x, y) = \exists z. (z \neq 0 \land x = y + (z * z)) $. \\
	
	% Bonus Task: \\
	% We need to write a formula $ gt_{\mathbb{Q}}(x, y) $ that is true precisely when $ x > y $ for any $ x, y \in \mathbb{Q} $. \\

	\textbf{(c)} \\
	We need to write a formula that is true over $\mathbb{N}$, but false over $\mathbb{R}$. \\
	Using the definitions of $ gt_{\mathbb{N}}(x, y) $ and $ gt_{\mathbb{R}}(x, y) $ from above,
	we can write the following formula:
	\begin{align*}
		\forall x. (x > 0 \lor x = 0)
	\end{align*}
	where $ x > 0 $ is $ gt_{\mathbb{N}}(x, 0) $ for $\mathbb{N}$ 
	and $ x > 0 $ is $ gt_{\mathbb{R}}(x, 0) $ for $\mathbb{R}$.
	This formula is true over $\mathbb{N}$ since for any $x \in \mathbb{N}$, $x$ is greater than or equal to $0$, which is not true for $\mathbb{R}$. \\
	% We can write a formula that is false in $\mathbb{R}$ due to its density property since $\mathbb{N}$ is not dense. \\
	% One such formula is $ \forall x. \forall y. ((x > y) \Rightarrow (\forall z. \: x = y + z \Rightarrow (z > 1 \lor z = 1))) $, \\ 
	% where $ x > y \equiv gt_{\mathbb{N}}(x, y) $ for $\mathbb{N}$ and $ x > y \equiv gt_{\mathbb{R}}(x, y) $ for $\mathbb{R}$. \\

    {\rule{17cm}{0.4pt}}
	\question[]
	\solutiontitle
	\textbf{(a)} \\
	Task 1: \\
	We are given the following formula over the logic $ ( \mathbb{R}, 0, 1, < ) $: \\
	\begin{align*}
		\forall z. ( l_1 < z \land z < u_1 \land l_2 < z \land z < u_2 \\
		\Rightarrow ( \exists w. \: l_1 < w \land w < u_1 \land l_2 < w \land w < u_2 \land w \neq z ) )
	\end{align*}

	Consider the inner formula that is quantified by $ \exists w $:
	\begin{align*}
		\exists w. ( l_1 < w \land w < u_1 \land l_2 < w \land w < u_2 \land w \neq z )
	\end{align*}

	Rewriting $ w \neq z $ as $ \neg (w = z) $, which is equivalent to $ w < z \lor w > z $, we get:
	\begin{align*}
		\exists w. ( l_1 < w \land w < u_1 \land l_2 < w \land w < u_2 \land (w < z \lor w > z) )
	\end{align*}
	Applying the distributive property and simplifying, we get the DNF form:
	\begin{align*}
		\exists w. ( l_1 < w \land w < u_1 \land l_2 < w \land w < u_2 \land w < z ) \\
		\lor \exists w. ( l_1 < w \land w < u_1 \land l_2 < w \land w < u_2 \land w > z )
	\end{align*}
	Consider the first disjunct: 
	\begin{align*}
		\exists w. ( l_1 < w \land w < u_1 \land l_2 < w \land w < u_2 \land w < z )
	\end{align*}
	In the context of the Q.E procedure for DLOWE, the lower bounds in this case are $ l_1, l_2 $ and the upper bounds are $ u_1, u_2, z $.
	Eliminating the existential quantifier, we get:
	\begin{align*}
		(l_1 < u_1 \land l_2 < u_1 \land l_1 < u_2 \land l_2 < u_2 \land l_1 < z \land l_2 < z)
	\end{align*}

	Consider the second disjunct:
	\begin{align*}
		\exists w. ( l_1 < w \land w < u_1 \land l_2 < w \land w < u_2 \land w > z )
	\end{align*}
	In the context of the Q.E procedure for DLOWE, the lower bounds in this case are $ l_1, l_2, z $ and the upper bounds are $ u_1, u_2 $.
	Eliminating the existential quantifier, we get:
	\begin{align*}
		(l_1 < u_1 \land l_2 < u_1 \land l_1 < u_2 \land l_2 < u_2 \land z < u_1 \land z < u_2)
	\end{align*}

	Combining the two disjuncts, we get:
	\begin{align*}
		(l_1 < u_1 \land l_2 < u_1 \land l_1 < u_2 \land l_2 < u_2 \land l_1 < z \land l_2 < z) \\
		\lor (l_1 < u_1 \land l_2 < u_1 \land l_1 < u_2 \land l_2 < u_2 \land z < u_1 \land z < u_2)
	\end{align*}
	Thus we have eliminated the existential quantifier $ \exists w $ from the original formula. \\

	Therefore, the original formula can be simplified to:
	\begin{align*}
		\forall z. ( l_1 < z \land z < u_1 \land l_2 < z \land z < u_2 \\
		\Rightarrow ((l_1 < u_1 \land l_2 < u_1 \land l_1 < u_2 \land l_2 < u_2 \land l_1 < z \land l_2 < z) \\
		\lor (l_1 < u_1 \land l_2 < u_1 \land l_1 < u_2 \land l_2 < u_2 \land z < u_1 \land z < u_2)) )
	\end{align*}

	Rewriting the implication as a disjunction, we get:
	\begin{align*}
		\forall z. \neg ( l_1 < z \land z < u_1 \land l_2 < z \land z < u_2 ) \\
		\lor ((l_1 < u_1 \land l_2 < u_1 \land l_1 < u_2 \land l_2 < u_2 \land l_1 < z \land l_2 < z) \\
		\lor (l_1 < u_1 \land l_2 < u_1 \land l_1 < u_2 \land l_2 < u_2 \land z < u_1 \land z < u_2))
	\end{align*}

	Rewriting the universal quantifier as an existential quantifier, and pushing the negation inside, we get:
	\begin{align*}
		\neg ( \exists z. ( l_1 < z \land z < u_1 \land l_2 < z \land z < u_2 ) ) \\
		\land \neg ((l_1 < u_1 \land l_2 < u_1 \land l_1 < u_2 \land l_2 < u_2 \land l_1 < z \land l_2 < z) \\
		\lor (l_1 < u_1 \land l_2 < u_1 \land l_1 < u_2 \land l_2 < u_2 \land z < u_1 \land z < u_2))
	\end{align*}

	Consider the inner existential formula:
	\begin{align*}
		\exists z. ( l_1 < z \land z < u_1 \land l_2 < z \land z < u_2 ) \\
		\land \neg ((l_1 < u_1 \land l_2 < u_1 \land l_1 < u_2 \land l_2 < u_2 \land l_1 < z \land l_2 < z) \\
		\lor (l_1 < u_1 \land l_2 < u_1 \land l_1 < u_2 \land l_2 < u_2 \land z < u_1 \land z < u_2))
	\end{align*}

	Applying De Morgan's law, we get:
	\begin{align*}
		\exists z. ( l_1 < z \land z < u_1 \land l_2 < z \land z < u_2 ) \\
		\land (\neg (l_1 < u_1 \land l_2 < u_1 \land l_1 < u_2 \land l_2 < u_2 \land l_1 < z \land l_2 < z) \\
		\land \neg (l_1 < u_1 \land l_2 < u_1 \land l_1 < u_2 \land l_2 < u_2 \land z < u_1 \land z < u_2))
	\end{align*}

	Consider the first inner negated formula:
	\begin{align*}
		\neg (l_1 < u_1 \land l_2 < u_1 \land l_1 < u_2 \land l_2 < u_2 \land l_1 < z \land l_2 < z)
	\end{align*}
	Using De Morgan's law, we get:
	\begin{align*}
		(l_1 > u_1 \lor l_1 = u_1) \lor (l_2 > u_1 \lor l_2 = u_1) \lor (l_1 > u_2 \lor l_1 = u_2) \lor (l_2 > u_2 \lor l_2 = u_2) \\
		\lor (l_1 > z \lor l_1 = z) \lor (l_2 > z \lor l_2 = z)
	\end{align*}

	Similarly for the other inner negated formula:
	\begin{align*}
		\neg (l_1 < u_1 \land l_2 < u_1 \land l_1 < u_2 \land l_2 < u_2 \land z < u_1 \land z < u_2)
	\end{align*}
	Applying De Morgan's law, we get:
	\begin{align*}
		(l_1 > u_1 \lor l_1 = u_1) \lor (l_2 > u_1 \lor l_2 = u_1) \lor (l_1 > u_2 \lor l_1 = u_2) \lor (l_2 > u_2 \lor l_2 = u_2) \\
		\lor (z > u_1 \lor z = u_1) \lor (z > u_2 \lor z = u_2)
	\end{align*}
	The conjunction of the above two formulae is:
	\begin{align*}
		((l_1 > u_1 \lor l_1 = u_1) \lor (l_2 > u_1 \lor l_2 = u_1) \lor (l_1 > u_2 \lor l_1 = u_2) \lor (l_2 > u_2 \lor l_2 = u_2) \\
		\lor (l_1 > z \lor l_1 = z) \lor (l_2 > z \lor l_2 = z)) \\
		\land \\
		((l_1 > u_1 \lor l_1 = u_1) \lor (l_2 > u_1 \lor l_2 = u_1) \lor (l_1 > u_2 \lor l_1 = u_2) \lor (l_2 > u_2 \lor l_2 = u_2) \\
		\lor (z > u_1 \lor z = u_1) \lor (z > u_2 \lor z = u_2))
	\end{align*}
	Applying the distributive property and simplifying, we get:
	\begin{align*}
		(l_1 > u_1 \lor l_1 = u_1) \lor (l_2 > u_1 \lor l_2 = u_1) \lor (l_1 > u_2 \lor l_1 = u_2) \lor (l_2 > u_2 \lor l_2 = u_2) \\
		\lor (( l_1 > z \lor l_1 = z) \land (z > u_1 \lor z = u_1)) \lor (( l_2 > z \lor l_2 = z) \land (z > u_1 \lor z = u_1)) \\
		\lor (( l_1 > z \lor l_1 = z) \land (z > u_2 \lor z = u_2)) \lor (( l_2 > z \lor l_2 = z) \land (z > u_2 \lor z = u_2))
	\end{align*}
	(The term after applying distributive property and before simplification is omitted for brevity). \\
	Putting the above back into the original formula, we get:
	\begin{align*}
		\exists z. ( l_1 < z \land z < u_1 \land l_2 < z \land z < u_2 ) \\
		\land (l_1 > u_1 \lor l_1 = u_1 \lor l_2 > u_1 \lor l_2 = u_1 \lor l_1 > u_2 \lor l_1 = u_2 \lor l_2 > u_2 \lor l_2 = u_2 \\
		\lor (( l_1 > z \lor l_1 = z) \land (z > u_1 \lor z = u_1)) \lor (( l_2 > z \lor l_2 = z) \land (z > u_1 \lor z = u_1)) \\
		\lor (( l_1 > z \lor l_1 = z) \land (z > u_2 \lor z = u_2)) \lor (( l_2 > z \lor l_2 = z) \land (z > u_2 \lor z = u_2)))
	\end{align*}
	Applying distributive property and simplifying, we get:
	\begin{align*}
		\exists z. ( l_1 < z \land z < u_1 \land l_2 < z \land z < u_2 \land l_1 > u_1 ) \\
		\lor \exists z. ( l_1 < z \land z < u_1 \land l_2 < z \land z < u_2 \land l_1 = u_1 ) \\
		\lor \exists z. ( l_1 < z \land z < u_1 \land l_2 < z \land z < u_2 \land l_2 > u_1 ) \\
		\lor \exists z. ( l_1 < z \land z < u_1 \land l_2 < z \land z < u_2 \land l_2 = u_1 ) \\
		\lor \exists z. ( l_1 < z \land z < u_1 \land l_2 < z \land z < u_2 \land l_1 > u_2 ) \\
		\lor \exists z. ( l_1 < z \land z < u_1 \land l_2 < z \land z < u_2 \land l_1 = u_2 ) \\
		\lor \exists z. ( l_1 < z \land z < u_1 \land l_2 < z \land z < u_2 \land l_2 > u_2 ) \\
		\lor \exists z. ( l_1 < z \land z < u_1 \land l_2 < z \land z < u_2 \land l_2 = u_2 ) \\
		\lor \exists z. ( l_1 < z \land z < u_1 \land l_2 < z \land z < u_2 \land ( l_1 > z \lor l_1 = z) \land (z > u_1 \lor z = u_1) ) \\
		\lor \exists z. ( l_1 < z \land z < u_1 \land l_2 < z \land z < u_2 \land ( l_2 > z \lor l_2 = z) \land (z > u_1 \lor z = u_1) ) \\
		\lor \exists z. ( l_1 < z \land z < u_1 \land l_2 < z \land z < u_2 \land ( l_1 > z \lor l_1 = z) \land (z > u_2 \lor z = u_2) ) \\
		\lor \exists z. ( l_1 < z \land z < u_1 \land l_2 < z \land z < u_2 \land ( l_2 > z \lor l_2 = z) \land (z > u_2 \lor z = u_2) )
	\end{align*}

	Simplifying the last four disjuncts, we get:
	\begin{align*}
		\exists z. ( l_1 < z \land z < u_1 \land l_2 < z \land z < u_2 \land l_1 > u_1 ) \\
		\lor \exists z. ( l_1 < z \land z < u_1 \land l_2 < z \land z < u_2 \land l_1 = u_1 ) \\
		\lor \exists z. ( l_1 < z \land z < u_1 \land l_2 < z \land z < u_2 \land l_2 > u_1 ) \\
		\lor \exists z. ( l_1 < z \land z < u_1 \land l_2 < z \land z < u_2 \land l_2 = u_1 ) \\
		\lor \exists z. ( l_1 < z \land z < u_1 \land l_2 < z \land z < u_2 \land l_1 > u_2 ) \\
		\lor \exists z. ( l_1 < z \land z < u_1 \land l_2 < z \land z < u_2 \land l_1 = u_2 ) \\
		\lor \exists z. ( l_1 < z \land z < u_1 \land l_2 < z \land z < u_2 \land l_2 > u_2 ) \\
		\lor \exists z. ( l_1 < z \land z < u_1 \land l_2 < z \land z < u_2 \land l_2 = u_2 ) \\
		\lor \exists z. ( l_1 < z \land z < u_1 \land l_2 < z \land z < u_2 \land l_1 > z \land z > u_1 ) \\
		\lor \exists z. ( l_1 < z \land z < u_1 \land l_2 < z \land z < u_2 \land l_2 > z \land z > u_1 ) \\
		\lor \exists z. ( l_1 < z \land z < u_1 \land l_2 < z \land z < u_2 \land l_1 > z \land z > u_2 ) \\
		\lor \exists z. ( l_1 < z \land z < u_1 \land l_2 < z \land z < u_2 \land l_2 > z \land z > u_2 ) \\
		\lor \exists z. ( l_1 < z \land z < u_1 \land l_2 < z \land z < u_2 \land l_1 > z \land z = u_1 ) \\
		\lor \exists z. ( l_1 < z \land z < u_1 \land l_2 < z \land z < u_2 \land l_1 = z \land z > u_1 ) \\
		\lor \exists z. ( l_1 < z \land z < u_1 \land l_2 < z \land z < u_2 \land l_1 = z \land z = u_1 ) \\
		\lor \exists z. ( l_1 < z \land z < u_1 \land l_2 < z \land z < u_2 \land l_2 > z \land z = u_1 ) \\
		\lor \exists z. ( l_1 < z \land z < u_1 \land l_2 < z \land z < u_2 \land l_2 = z \land z > u_1 ) \\
		\lor \exists z. ( l_1 < z \land z < u_1 \land l_2 < z \land z < u_2 \land l_2 = z \land z = u_1 ) \\
		\lor \exists z. ( l_1 < z \land z < u_1 \land l_2 < z \land z < u_2 \land l_1 > z \land z = u_2 ) \\
		\lor \exists z. ( l_1 < z \land z < u_1 \land l_2 < z \land z < u_2 \land l_1 = z \land z > u_2 ) \\
		\lor \exists z. ( l_1 < z \land z < u_1 \land l_2 < z \land z < u_2 \land l_1 = z \land z = u_2 ) \\
		\lor \exists z. ( l_1 < z \land z < u_1 \land l_2 < z \land z < u_2 \land l_2 > z \land z = u_2 ) \\
		\lor \exists z. ( l_1 < z \land z < u_1 \land l_2 < z \land z < u_2 \land l_2 = z \land z > u_2 ) \\
		\lor \exists z. ( l_1 < z \land z < u_1 \land l_2 < z \land z < u_2 \land l_2 = z \land z = u_2 )
	\end{align*}

	The following terms can be pulled out from the first eight disjuncts since they do not refer to $ z $:
	\begin{align*}
		(l_1 > u_1) \lor (l_1 = u_1) \lor (l_2 > u_1) \lor (l_2 = u_1) \lor (l_1 > u_2) \lor (l_1 = u_2) \lor (l_2 > u_2) \lor (l_2 = u_2)
	\end{align*}

	Upon removing the above terms, the first eight disjuncts all become the same:
	\begin{align*}
		\exists z. ( l_1 < z \land z < u_1 \land l_2 < z \land z < u_2 )
	\end{align*}
	The existential can be eliminated using the Q.E procedure for DLOWE, and we get:
	\begin{align*}
		l_1 < u_1 \land l_1 < u_2 \land l_2 < u_1 \land l_2 < u_2
	\end{align*}
	Combining this formula with the terms that were pulled out, we get the simplified formula:
	\begin{align*}
		((l_1 > u_1 \lor l_1 = u_1) \lor (l_2 > u_1 \lor l_2 = u_1) \lor (l_1 > u_2 \lor l_1 = u_2) \lor (l_2 > u_2 \lor l_2 = u_2)) \\
		\land (l_1 < u_1 \land l_1 < u_2 \land l_2 < u_1 \land l_2 < u_2)
	\end{align*}

	The next four disjuncts are:
	\begin{align*}
		\exists z. ( l_1 < z \land z < u_1 \land l_2 < z \land z < u_2 \land l_1 > z \land z > u_1 ) \\
		\lor \exists z. ( l_1 < z \land z < u_1 \land l_2 < z \land z < u_2 \land l_2 > z \land z > u_1 ) \\
		\lor \exists z. ( l_1 < z \land z < u_1 \land l_2 < z \land z < u_2 \land l_1 > z \land z > u_2 ) \\
		\lor \exists z. ( l_1 < z \land z < u_1 \land l_2 < z \land z < u_2 \land l_2 > z \land z > u_2 ) \\
	\end{align*}
	and they can be eliminated since they contain terms that are not satisfiable \\
	(e.g. $ l_1 < z \land l_1 > z $ which cannot be satisfied). \\
	The next twelve disjuncts contain terms of the form $ l_1 = z $ and also $ l_1 < z $ or $ l_1 > z $ which are not satisfiable together. \\
	
	Bringing back the outermost negation, we get the simplified formula:
	\begin{align*}
		\neg 
		((l_1 > u_1 \lor l_1 = u_1) \lor (l_2 > u_1 \lor l_2 = u_1) \lor (l_1 > u_2 \lor l_1 = u_2) \lor (l_2 > u_2 \lor l_2 = u_2)) \\
		\land (l_1 < u_1 \land l_1 < u_2 \land l_2 < u_1 \land l_2 < u_2)
	\end{align*}
	Simplifying the above formula (and using the fact that $ \neg (l_1 > u_1 \lor l_1 = u_1) \equiv l_1 < u_1 $), we get:
	\begin{align*}
		(l_1 < u_1 \land l_1 < u_2 \land l_2 < u_1 \land l_2 < u_2) \\
		\lor \neg (l_1 < u_1 \land l_1 < u_2 \land l_2 < u_1 \land l_2 < u_2)
	\end{align*}
	which is a equivalent to the original formula. \\	
	We can also see that the formula is a tautology of the form $ A \lor \neg A $, where $ A $ is $ l_1 < u_1 \land l_1 < u_2 \land l_2 < u_1 \land l_2 < u_2 $. \\
	Hence, the formula simplifies to $ \top $. \\

	Task 2: \\
	Encoding the above formula in Z3 and using the Q.E tactic:
	\begin{python}
		from z3 import *

l1 = Real("l1")
l2 = Real("l2")
u1 = Real("u1")
u2 = Real("u2")
w = Real("w")
z = Real("z")

goal = Goal()
goal.add(
    ForAll(
        [z],
        Implies(
            And(l1 < z, z < u1, l2 < z, z < u2),
            Exists([w], And(l1 < w, w < u1, l2 < w, w < u2, w != z)),
        ),
    )
)

tactic = Tactic("qe")
result = tactic(goal)
print(result)
	\end{python}
	The output of the above code is:
	\begin{python}
[[]]
	\end{python}

	\textbf{(b)} \\
	Task 1: \\
	Let us name the vertices of the given graph as $ A, B, C, D $ where A is the topmost vertex and D is the bottommost vertex. \\
	C is the vertex on the left and B is the vertex on the right. \\
	The following vertices are connected by an edge: A and B, A and C, B and D, C and D. \\
	Let us represent each vertex as an interval, using $ (l_{A}, u_{A}) $ to represent vertex A, and so on. \\
	If G is an interval graph, for two intervals $ (l_{A}, u_{A}) $ and $ (l_{B}, u_{B}) $ to be connected by an edge
	there must exist a $ z \in \mathbb{R} $ such that $ l_{A} < z < u_{A} $ and $ l_{B} < z < u_{B} $. \\
	Written as a formula, this is:
	\begin{align*}
		\exists z. (l_{A} < z \land z < u_{A}) \land (l_{B} < z \land z < u_{B})
	\end{align*}
	For the given graph, the conjunction of the following formulae must be true (for it to be an interval graph):
	\begin{align*}
		\exists z. (l_{A} < z \land z < u_{A}) \land (l_{B} < z \land z < u_{B}) \\
		\land \exists z_1. (l_{A} < z_1 \land z_1 < u_{A}) \land (l_{C} < z_1 \land z_1 < u_{C}) \\
		\land \exists z_2. (l_{B} < z_2 \land z_2 < u_{B}) \land (l_{D} < z_2 \land z_2 < u_{D}) \\
		\land \exists z_3. (l_{C} < z_3 \land z_3 < u_{C}) \land (l_{D} < z_3 \land z_3 < u_{D})
	\end{align*}
	Each existential formula represents an edge in the graph. \\
	We also need to ensure that if two vertices are not connected by an edge, then their intervals do not intersect. \\
	We can state the above as:
	\begin{align*}
		\neg (\exists z. (l_{A} < z \land z < u_{A}) \land (l_{D} < z \land z < u_{D})) \\
		\land \neg (\exists z. (l_{B} < z \land z < u_{B}) \land (l_{C} < z \land z < u_{C}))
	\end{align*}
	Combining the above formulae, we get the required formula:
	\begin{align*}
		\exists z. (l_{A} < z \land z < u_{A}) \land (l_{B} < z \land z < u_{B}) \\
		\land \exists z_1. (l_{A} < z_1 \land z_1 < u_{A}) \land (l_{C} < z_1 \land z_1 < u_{C}) \\
		\land \exists z_2. (l_{B} < z_2 \land z_2 < u_{B}) \land (l_{D} < z_2 \land z_2 < u_{D}) \\
		\land \exists z_3. (l_{C} < z_3 \land z_3 < u_{C}) \land (l_{D} < z_3 \land z_3 < u_{D}) \\
		\land \neg (\exists z_4. (l_{A} < z_4 \land z_4 < u_{A}) \land (l_{D} < z_4 \land z_4 < u_{D})) \\
		\land \neg (\exists z_5. (l_{B} < z_5 \land z_5 < u_{B}) \land (l_{C} < z_5 \land z_5 < u_{C}))
	\end{align*}
	
	Task 2: \\
	Encoding the above formula in Z3:
	\begin{python}
from z3 import *

la = Real("la")
ua = Real("ua")
lb = Real("lb")
ub = Real("ub")
lc = Real("lc")
uc = Real("uc")
ld = Real("ld")
ud = Real("ud")
z = Real("z")
z1 = Real("z1")
z2 = Real("z2")
z3 = Real("z3")
z4 = Real("z4")
z5 = Real("z5")

s = Solver()
s.add(
	And(
		Exists([z], And(la < z, z < ua, lb < z, z < ub)),
		Exists([z1], And(la < z1, z1 < ua, lc < z1, z1 < uc)),
		Exists([z2], And(ld < z2, z2 < ud, lb < z2, z2 < ub)),
		Exists([z3], And(ld < z3, z3 < ud, lc < z3, z3 < uc)),
		Not(Exists([z4], And(la < z4, z4 < ua, ld < z4, z4 < ud))),
		Not(Exists([z5], And(lb < z5, z5 < ub, lc < z5, z5 < uc))),
	)
)
print(s.check())
	\end{python}
	The output of the above code is:
	\begin{python}
unsat
	\end{python}
		
	
    {\rule{17cm}{0.4pt}}
	\question[]
	\solutiontitle


    {\rule{17cm}{0.4pt}}

\end{questions}
\end{document}